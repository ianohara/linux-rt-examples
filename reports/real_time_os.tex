%%%%%%%%%%%%%%%%%%%%%%%%%%%%%%%%%%%%%%%%%
% Laboratory Report LaTeX Template
%
% This template has been downloaded from:
% http://www.latextemplates.com
%
% Original header:
%
% This is a LaTeX version of the sample laboratory report
% from Virginia Tech's copyrighted 08-09 CHEM 1045/1046 lab manual.
% Reproduction of this one appendix section for academic purposes
% should fall under fair use.
%
%%%%%%%%%%%%%%%%%%%%%%%%%%%%%%%%%%%%%%%%%

%----------------------------------------------------------------------------------------
%	DOCUMENT CONFIGURATIONS
%----------------------------------------------------------------------------------------

\documentclass{article}

\title{Overview of Real Time Operating Systems \\ Fall 2012 Independent Study} % Title

\author{Ian \textsc{O'Hara}} % Author name

\begin{document}

\maketitle % Insert the title, author and date

\begin{tabular}{lr}
2012/09/20 & At UPenn's Modlab (GRASP Subsidiary)\\ % Date the experiment was performed and partner's name
Advisor: Dr. Mark Yim % Instructor/supervisor
\end{tabular}

\setlength\parindent{0pt} % Removes all indentation from paragraphs

\renewcommand{\labelenumi}{\alph{enumi}.} % Make numbering in the enumerate environment by letter rather than number (e.g. section 6)

\section{DEFINITIONS}
The basic terminology of real time systems is taken from \cite{Laplante}.  Unless otherwise noted, assume these definitions can be found there.
\begin{description}
\item[System]{A mapping of a set of inputs into a set of outputs}
\item[Response Time]{The time a system takes to map a set of inputs to the corresponding set of outputs}
\item[Real Time System (RTS)]{A system where there is a bound set on the response time of a system.  When this bound is not met, the system is considered in a failed state.}
\item[Hard Real Time System (HRTS)]{A RTS in which failure to meet the response time bound leads to catastrophic failure.  IE: Response times must be deterministically met.}
\item[Soft Real Time System (SRTS)]{A RTS in which the response time bounds must be met the majority of the time. IE: Performance is degraded, but not destroyed, when response times are not met.}
\item[Process Preemption]{A process preemption is the suspension of a process so that another higher priority process can run.  This can occur for a number of reasons, and is essentil in RTS.}
\end{description}

It is importantant to note that while RTS often deal with ``fast" times, a RTS does not need to be fast.  It simply needs to have explicit bounds on response times.

Another phrasing from \cite{RealTimeLinux} is that a Real Time Application is one in which there are operational deadlines between some event being triggered and the applications response to that event.  The use of a Real Time Operating System (RTOS) gives the programmer calculated (hard real time) or measurement predicted (softer real time) response times.

\section{INTRODUCTION}
%------------------------------------------------------------------------------
%	BIBLIOGRAPHY
%------------------------------------------------------------------------------

\section{REAL TIME LINUX}
The traditional linux kernel allows one process to preempt other processes in limited cases.  Specifically, preemption is controlled by the kernel and occurs only when:

\subsection{Development Info}
It looks like \cite{Osdl} is highly involved with the Real Time Linux development process.  Their Real Time Linux project page \cite{OsdlRealTimeLinux} cites both Ingo Molnar and Thomas Gleixner as being the lead developers.

The OSDL site cites a few important locations for downloading RTL requirements:
\begin{enumerate}
\item The Linux Kernel at kernel.org
\item The Real Time Preempt Patch at \cite{RealTimeLinuxPatch}
\end{enumerate}
Their precompiled kernels are debian based, which is nice.  That is what I want anyway.
\begin{enumerate}
\item User-Mode code is running (Kernel preempts it)
\item Kernel returns from a system call or an interrupt back to user space
\item Kernel code blocks on a mutex or explicitly yields
\end{enumerate}
These points are from the FAQ at \cite{RealTimeLinux}.

\section{NOTES ON USE}
Real time software can't have it memory paged out, so in linux a call to {\texttt mlockall()} needs to be made.  This makes sure all of a program's memory stays in RAM. (FAQ at \cite{RealTimeLinux})

\begin{thebibliography}{9}

\bibitem{Laplante}
Laplante, Phillip A. Real Time System Design and Analysis, 3rd Ed. Pscataway, NJ: IEEE Press, 2004. Print.
\bibitem{RealTimeLinux}
Real-Time Linux Wiki, https://rt.wiki.kernel.org/index.php/Main\_Page, 2012-09-18
\bibitem{BuildEmbedLinux}
Yaghmour, K.; Masters, J; Ben-Yossef, G; and Gerem, P. Building Embedded Linux Systems, 2nd Ed. Sebastopol, CA: O'Reilly Media, Inc. 2008. Print.
\bibitem{RTL3Announce}
Gleixner, T. ``[ANNOUNCE] 3.0-rc7-rt0". https://lkml.org/lkml/2011/7/19/309. Email
\bibitem{RTL3Announce2}
http://www.h-online.com/open/features/Kernel-Log-real-time-kernel-goes-Linux-3-0-1382791.html
\bibitem{Osadl}
Open Source Software for Automation and Other Industries. https://www.osadl.org/Home.1.0.html
\bibitem{OsadlRealTimeLinux}
OSDL Real Time Linux Project, https://www.osadl.org/Realtime-Linux.projects-realtime-linux.0.html
\bibitem{RealTimeLinuxPatch}
Real Time Linux Latest Patches, http://www.kernel.org/pub/linux/kernel/projects/rt/
\bibitem{RealTimeLinuxInstall}
Real Time Linux Installation, https://www.osadl.org/Realtime-Preempt-Kernel.kernel-rt.0.html
\end{thebibliography}

\end{document}
